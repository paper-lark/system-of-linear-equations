\documentclass[12pt, a4paper]{report}

% Language configuration
\usepackage{amsmath,amsthm,amssymb}
\usepackage{mathtext}
\usepackage[T1,T2A]{fontenc}
\usepackage[utf8]{inputenc}
\usepackage[english,russian]{babel}

% Page config
\usepackage[paperwidth=21cm, paperheight=29.7cm, top=25mm, bottom=25mm, left=25mm, right=25mm]{geometry}
\usepackage{setspace}
\usepackage{enumitem}
\setlist[description]{leftmargin=\parindent, labelindent=\parindent}

% Source code integration
\usepackage{listings}
\lstdefinestyle{CStyle}{
        numberstyle=\tiny,
        basicstyle=\footnotesize,
        breakatwhitespace=false,         
        breaklines=true,                 
        captionpos=b,                    
        keepspaces=true,                 
        numbers=left,                    
        numbersep=5pt,                  
        showspaces=false,                
        showstringspaces=false,
        showtabs=false,                  
        tabsize=2,
        language=C
}
\lstset{language=C,style=CStyle}

% Matrix environment
\newenvironment{amatrix}[1]{%
        \left(\begin{array}{@{}*{#1}{c}|c@{}}
}{%
        \end{array}\right)
}

% Main info
\title{Отчет по заданию \#1 по курсу "Введение в численные методы"}
\author{Журавский Максим \\
        курс II, группа 203}
\date{2017}

\begin{document}



% Title page
\begin{titlepage}
    \begin{center}
    {\scshape\large Московский Государсвенный Университет \\
                    имени М.В.Ломоносова \par}
    {\scshape\large Факультет Вычислительной математики и Кибернетики \par}
    
    \begin{center}
    \line(1,0){400}
    \end{center}
    
    \vspace{6cm}
	{\Large\bfseries\setstretch{1.25} Компьютерный практикум по учебному курсу
                     "Введение в численные методы" \\
                     Задание \textnumero 1 \par}
    \vspace{3cm}
    {\Large\bfseries Отчет \\
            о выполненном задании \par}
    {\large студента 203 учебной группы факультета ВМК МГУ \\
            Журавского Максима Игоревича \par}
    \vfill
    {\normalsize Москва, 2017 \par}
    \end{center}
\end{titlepage}



% Table of contents
\tableofcontents
\newpage



% Introduction
\chapter{Введение}
Целью данной работы является реализация численный методов нахождения:

\begin{itemize}
\item{решения заданных систем линейных алгебраических уравнений методами Гаусса,
        в т.ч. методом Гаусса с выбором главного элемента, и методом верхней релаксации.}
\item{определителя заданных матриц}
\item{матрицы, обратной данной}
\end{itemize}
\normalsize{Кроме того, в работе исследуется вопросы устойчивости метода Гаусса
            при больших размерах матрицы коэффициентов и скорости сходимости
            к точному решению задачи итераций метода верхней релаксации при изменении
            итерационного параметра $\omega$}.
\newpage



% Theory
\chapter{Алгоритмы решения}

\section{Метод Гаусса}
\normalsize{Решения системы линейных алгебраических уравнений методом Гаусса
            производится в два этапа, именуемые "прямым" и "обратным" ходом,
            в результате которых из соответствующей исследуемой системе линейных
            алгебраических уравнений расширенной матрицы находится решение системы.}
\normalsize{В результате "прямого хода" матрица коэффициентов, входящая в расширенную матрицу,
            путем линейных пребразований строк и их перестановок 
            последовательно приводится к верхнему треугольному виду.
            На этапе "обратного хода" выполняется восстановление решения путем
            прохода по строкам матрицы в обратном направлении.}

\normalsize{В отличие от классического метода Гаусса в метода Гаусса с выбором главного элемента
            на каждой итерации выбирается максимальный из всех элементов матрицы, что позволяет
            уменьшить погрешность вычислений.
            Более подробно оба метода описаны в \cite{computational_science_gauss}.

\normalsize{Задача вычисления определителя сводится к задаче
            приведения исследуемой матрицы к верхнему треугольному виду
            и последующим вычислением произведения её диагональных элементов. 
            Для приведения матрицы к диагональному виду используется классический метод Гаусса.}

\normalsize{Задача нахождения обратной матрицы решается с помощью метода Гаусса-Жордана. 
            Над дополненной матрицей, составленной из столбцов исходной матрицы и единичной
            того же порядка, производятся преобразования метода Гаусса, в результате которых
            исходная матрица принимает вид единичной матрицы, а на месте единичной образуется
            матрица, обратная исходной. Теоретическое обоснование может быть найдено в \cite{algebra}}.

\section{Метод верхней релаксации}

\normalsize{Метод вехний релакцации является стационарным итерационным методом, 
            в котором каждый следующий вектор приближения вычисляется по формуле:
            \[ (D + \omega T_н) \frac{y_{k+1} - y_k}{\omega} + A y_k = f \]
            где 
            \begin{description}
                \item[$\omega$] - итерационный параметр, \\
                \item[$y_k$] - $k$-й вектор приближения, \\
                \item[$y_{k+1}$] - ($k+1$)-й вектор приближения, \\
                \item[$A$] - матрица коэффициентов, \\
                \item[$f$] - правая часть системы, \\
                \item[$T_н$] - нижняя диагональная матрица матрицы $A$, \\
                \item[$D$] - матрица диагональных элементов матрицы $A$. \\
            \end{description}}

\normalsize{По теореме Самарского \cite{computational_science_iter} для положительно определенных матриц $A$
            метод верхней релаксации сходится, если $0 < \omega < 2$. Однако, оптимальное значение $\omega$ в каждом случае
            находится экспериментально и зависит от матрицы $A$. В данной реализации метода допускается возможность
            изменения итерационного параметра с помощью флагов компиляции. Значением по умолчанию является $\omega = \frac{4}{3}$.}
\newpage



% Program
\chapter{Программа}
\normalsize{В качестве языка программорования системы был выбран язык C ввиду его гибкости и высоко производительности. 
            Программа реализована модульно. Основная часть программы реализована в файле \textbf{main.c}.
            Исходный код модулей, отвечающих за реализацию метода Гаусса, метода Гаусса с выбором главного члена и
            метода верхней релаксации, расположен в файлах \textbf{gauss.c}, \textbf{modified-gauss.c} и \textbf{iteration.c} соответственно.
            Функции из приложения \textnumero 2 находятся в файле \textbf{functions.c}.
            Ниже приведено содержание каждого из файлов.}

\vspace{1cm}
\normalsize{\bfseries main.c:}
\lstinputlisting{../src/main.c}

\vspace{1cm}
\normalsize{\bfseries gauss.c:}
\lstinputlisting{../src/gauss.c}

\vspace{1cm}
\normalsize{\bfseries modified-gauss.c:}
\lstinputlisting{../src/modified-gauss.c}

\vspace{1cm}
\normalsize{\bfseries iteration.c:}
\lstinputlisting{../src/iteration.c}

\vspace{1cm}
\normalsize{\bfseries functions.c:}
\lstinputlisting{../src/functions.c}

\newpage



% Testing
\chapter{Тестирование}
\normalsize{Тестирование правильности решения систем линейных алгебраических уравнений программой
            проводилось как с помощью проверки результатов работы программы путем подстановки их 
            в исходную систему, так и с помощью сравнения результат работ двух методов. Тестирование
            нахождения обратных матриц и определителя проверялось вручную для матриц, размером меньше
            $3 x 3$, и с помощью пакета Maple для матриц больше порядка.}
\normalsize{Ниже приведены условия тестов в виде расширенных матриц систем алгебраических уравнений и результаты работы программы.
            Первые три теста взяты из варианта \textnumero 13 тестовых заданий.}

% Test #1
\vspace{1cm}
\large{Тест \#1}
\[\begin{amatrix}{4}
        3 & -2 &  2 & -2 &  8 \\
        2 & -1 &  2 &  0 &  4 \\
        2 &  1 &  4 &  8 & -1 \\
        1 &  3 & -6 &  2 &  3 \\
\end{amatrix}\]

\normalsize{Результат работы метода Гаусса: \par}
\normalsize{> determinant: \par
            24 \par
            > inverse matrix: \par
            -0.5     1.333333333   -0.1666666667    0.1666666667 \par
              -5     8.666666667    -1.333333333    0.3333333333 \par
              -2             3.5            -0.5               0 \par
            1.75    -3.166666667    0.5833333333  -0.08333333333 \par
            > solution: \par
            2              -3            -1.5             0.5 \par}

\vspace{0.25cm}
\normalsize{Результат работы метода Гаусса с выбором главного элемента: \par}
\normalsize{> solution: \par
            2              -3            -1.5             0.5 \par}
\vspace{0.25cm}
\normalsize{Результат работы метода верхней релаксации: \par}
\normalsize{> log: 235 iterations made \par
            > solution: \par
            -nan              -nan            -nan             -nan \par}
\vspace{1cm}

% Test #2
\large{Тест \#2}
\[\begin{amatrix}{4}
        2 &  3 &  1 &  2 &  4 \\
        4 &  3 &  1 &  1 &  5 \\
        1 & -7 & -1 & -2 &  7 \\
        2 &  5 &  1 &  1 &  1 \\
\end{amatrix}\]

\normalsize{Результат работы метода Гаусса: \par}
\normalsize{> determinant: \par
            2 \par
            > inverse matrix: \par
                        -1               2              -1              -2 \par
                        -1             1.5              -1            -1.5 \par
                         8           -14.5               9            16.5 \par
                        -1               3              -2              -4 \par
            > solution: \par
                        -3              -5              39              -7 \par
            }

\vspace{0.25cm}
\normalsize{Результат работы метода Гаусса с выбором главного элемента: \par}
\normalsize{> solution: \par
            -3              -5            39             -7 \par}
\vspace{0.25cm}
\normalsize{Результат работы метода верхней релаксации: \par}
\normalsize{> log: 271 iterations made \par
            > solution: \par
            -nan              -nan            -nan             -nan \par}
\vspace{1cm}

% Test #3
\large{Тест \#3}
\[\begin{amatrix}{4}
        1 & -1 &  1 & -1 & 0 \\
        4 & -1 &  0 & -1 & 0 \\
        2 &  1 & -2 &  1 & 0 \\
        5 &  1 &  0 & -4 & 0 \\
\end{amatrix}\]

\normalsize{Результат работы метода Гаусса: \par}
\normalsize{> determinant:
            0
            > inverse matrix:
            does not exist
            > solution:
            cannot be found using Gaussian elimination method}
\vspace{1cm}

% Test #4
\large{Тест \#4}
\[\begin{amatrix}{3}
        1 & 0 & 1 & 2 \\
        0 & 1 & 0 & 3 \\
        1 & 0 & 0 & 4 \\
\end{amatrix}\]

\normalsize{Результат работы метода Гаусса: \par}
\normalsize{> determinant: \par
            -1 \par
            > inverse matrix: \par
                         0               0               1 \par
                         0               1               0 \par
                         1               0              -1 \par
            > solution: \par
                         4               3              -2 \par}

\vspace{0.25cm}
\normalsize{Результат работы метода Гаусса с выбором главного элемента: \par}
\normalsize{> solution: \par
            4               3              -2 \par}
\vspace{0.25cm}
\normalsize{Результат работы метода верхней релаксации: \par}
\normalsize{> log: 1 iterations made \par
            > solution: \par
            2.6666666667    4              -nan \par}
\vspace{1cm}

% Test #5
\large{Тест \#5}
\[\begin{amatrix}{2}
        1 & 1 & 0 \\
        1 & 2 & 1 \\
\end{amatrix}\]

\normalsize{Результат работы метода Гаусса: \par}
\normalsize{> determinant: \par
            1 \par
            > inverse matrix: \par
                         2              -1 \par
                        -1               1 \par
            > solution: \par
                        -1               1 \par}
\vspace{0.25cm}
\normalsize{Результат работы метода Гаусса с выбором главного элемента: \par}
\normalsize{> solution: \par
            -1              1 \par}
\vspace{0.25cm}
\normalsize{Результат работы метода верхней релаксации: \par}
\normalsize{> log: 19 iterations made \par
            > solution: \par
            -1.000000003     1.000000001 \par}
\vspace{1cm}

% Test #6
\large{Тест \#6}
\[\begin{amatrix}{3}
        3 & -1 &  1 & 3 \\
        1 &  1 &  1 & 5 \\
        4 & -1 &  4 & 5 \\
\end{amatrix}\]

\normalsize{Результат работы метода Гаусса: \par}
\normalsize{> determinant: \par
            10 \par
            > inverse matrix: \par
                       0.5             0.3            -0.2 \par
                         0             0.8            -0.2 \par
                      -0.5            -0.1             0.4 \par
            > solution: \par
                         2               3               0 \par}

\vspace{0.25cm}
\normalsize{Результат работы метода Гаусса с выбором главного элемента: \par}
\normalsize{> solution: \par
            2              3            0 \par}
\vspace{0.25cm}
\normalsize{Результат работы метода верхней релаксации: \par}
\normalsize{> log: 25 iterations made \par
            > solution: \par
            2.000000001             3.000000001            0     \par}
\vspace{1cm}

% Test #7
\large{Тест \#7}
\[\begin{amatrix}{3}
        2 & -1 &  1 &  6 \\
        1 &  2 & -1 & -1 \\
        1 & -1 &  2 & 11 \\
\end{amatrix}\]

\normalsize{Результат работы метода Гаусса: \par}
\normalsize{> determinant: \par
            6 \par
            > inverse matrix: \par
                       0.5    0.1666666667   -0.1666666667 \par
                      -0.5             0.5             0.5 \par
                      -0.5    0.1666666667    0.8333333333 \par
            > solution: \par
                         1               2               6 \par}
\vspace{0.25cm}
\normalsize{Результат работы метода Гаусса с выбором главного элемента: \par}
\normalsize{> solution: \par
            1              2            6 \par}
\vspace{0.25cm}
\normalsize{Результат работы метода верхней релаксации: \par}
\normalsize{> log: 58 iterations made \par
            > solution: \par
            0.9999999987     2.000000002     6.000000002 \par}
\vspace{1cm}
            
% Ending
\normalsize{Ввиду того, что лишь в последних двух тестах матрицы были положительно определенными,
            результаты работы метода верхней релаксации в большинстве других были далеки от истинных.
            В целом, при $\omega = \frac{4}{3}$ метод верхней релаксации получал решение, 
            достаточно близкое к точному менее, чем за 100 итераций.}            
\newpage

% Conclusion
\chapter{Заключение}
\normalsize{В ходе работы мной были освоены методы решения систем линейных алгебраических уравнений, 
            а именно: метод верхней релаксации и метод Гаусса. Из результатов легко убедиться в том,
            что ввиду ограниченной применимости метода верхней релаксации, порой он оказывается неприменимым.
            Тем не менее, при аккуратном подборе итерационного параментра, он может оказаться даже быстрей
            метода Гаусса. Преимуществом метода Гаусса же является простота его реализиции и широкая применимость
            в задачах, связанных с исследованием матриц, в том числе нахождением обратной матрицы и детерминанта.}
\newpage


% Reference
\begin{thebibliography}{}
    \bibitem{computational_science_gauss}Самарский А.А. Введение в численные методы. СПб., 2009. С. 91.
    \bibitem{algebra} Белоусов И.В. Матрицы и определители. Кишинев, 2006. С. 68.
    \bibitem{computational_science_iter}Самарский А.А. Введение в численные методы. СПб., 2009. С. 102.
\end{thebibliography}

\end{document}